\documentclass[a4paper,12pt]{article} % format of document

\usepackage[english,russian]{babel} % add eng,rus(base) package
\usepackage[T1,T2A]{fontenc}        % add eng,rus encoding support
\usepackage[utf8]{inputenc}         % add UTF8 support
\usepackage{soul}         	    % add l a t e r s p a c i n g
\usepackage{longtable}         	    % To display tables on several pages
\usepackage{booktabs}         	    % For pretter tables
\usepackage{enumitem}         	    % advanced list support

\usepackage{amsmath, amsfonts, amssymb, amsthm, mathtools} % add math support

\usepackage{geometry}  % add document's fields correction support
\geometry{top=25mm}    % top field
\geometry{bottom=30mm} % bottom field
\geometry{left=20mm}   % left field
\geometry{right=20mm}  % right field

\linespread{1}               % length between str
\setlength{\parindent}{20pt} % red str
\setlength{\parskip}{12pt}   % length between paragraphs

\usepackage[backend=biber, style=authoryear-icomp]{biblatex} % add bibliography support
\addbibresource{$HOME/latex-templates/biblio.bib}            % path to bibliography base
\usepackage{csquotes}                                        % advanced facilities for inline and display quotations

\usepackage{indentfirst} % first paragraph with red str

% Must be the last command into the preamble of document.
\usepackage{hyperref} % All references in document turn into hyperlinks
\hypersetup{
unicode=true,      % юникод в названиях разделов pdf
colorlinks=true,   % цветные ссылки вместо ссылок в рамках
linkcolor=blue,    % внутренние ссылки
citecolor=green,   % ссылки на библиографию
filecolor=magneta, % ссылки на файлы
urlcolor=blue,     % ссылки на url
}
 % here is document's settings for russian
%\input{$HOME/studyproject/universe/history/preamble-beamer-eng.tex} % here is document's settings for english


\title{Историческое сочинение по теме: Праздники в период царствования Елизоветы Петровны}
\author{Немков Николай Максимович СМ4-11}

\date{28.11.2023}

\begin{document}

\maketitle
\begin{center}
Московски Государственный Технический Университет им. Н.Э. Баумана

Преподаватель: Щербакова Ольга Михайловна
\end{center}

\newpage
\large{
	В жанровый арсенал развлекательной культкры елизаветинского Петербургра входили: триумфальные шествия, обеды, баллы-маскарады, фейерверки и иллюминации, театральные постановки и музыкальные концерты, а также всевозможные барские затеии и забавы.

	Офицальный ритуал светских праздников сложился с петровских времен. Елизавета внимательно следила за всеми европейскими новшествами. Непременными атрибутами офицальных праздников являлись \textbf{литургия} в придворной церкви, дополняемая перезвоном колоколов городских церквей, \textbf{парад и угощения} стоящих в параде офицеров, а также \textbf{дармовое угощение} для простолюдинов. Далее следовал торжественный прасдничный \textbf{обед} в Зимнем дворце. Обед сменялся \textbf{балом или маскарадом}. Все праздники сопровождались фейерверком и салютом, а каронационные торжества, помимо прочего --- \textbf{триумфальным шествием} императрицы и ее свиты через специально построенные для этого праздника триумфальные ворота.

	Праздничные обеды паражали своими масштабами. они продолжались четыре-семь часов. В заранее утвержденное меню входили "кушанья всех возможных стран Европы". Лебеди, поросята, дичь, говядина и прочие "разносолы" сменяли друг друга непрерывно. За угощением пили венгерское, бургундское, шампанское, пиво английское, белое красное вино, по две и более бутылки на человека. Главным тостом было за здоровье и долголетие "Ея Императорского Величества". В момент испития батарея стоявшая у дворца, по взмаху платка служителя непрерывно палила --- тосты следовали один за другим.

	Во времена Елизаветы Петровны придворные балы достигают огромного размаха. На них собираются до трех тысяч человек. Балы и маскарады были не только развлечением, но и повинностью: явка гостей была обязательной. Особое место в развлекательной "политике" Елизоветы занимали маскарады. Они устраивались два раза в неделю. Для "масок" выставляли напитки и закуски, ставились карточные столы и разыгрывались лотереи. Елизавета блистала красотой и нарядами. Так называемое чудачества Императрицы внесли свою лепту в маскарады с переодеванием --- дамы переодевались в мужские костюмы, а мужчины в дамские. Елизавета задавала тон и была законодательницей мод. Ее гардероб включал пятнадцать тысяч платьев и не одно из них царица не надела дважды.

	Среди зрелищ XVIII в. иллюминация и фейерверки --- самые распространенные и самые массовые. "В России порохом дорожат не более чем песком" -- удивлялись иностранцы роскоши русских фейерверков. Для иллюминации использовались зажженные глиняные плошки, наполнение горяжьим салом. Особой красотой отличалась иллюминация в Петергофе, чему способствовала круговая подсветка фонтанов под "итольянскую голосную и инструментальную музыку". Фейерверк становится символом государственной власти при Петре I. В елизаветенское время он продолжал оставаться одним из ведущих жанров в развлекательной культуре. Праздничный фейерверк заканчивался красочным салютом из сотен орудий цитаделей и фортов Кронштадта, бастионов Адмиралтейской и Петропавловской крепостей и стоявших на Неве кораблей.

	Не прошли бесследно годы ночной, неумеренной жизни, отсутствие всяческих ограничений в еде, питье и развлечениях. Невероятно, но даже смерть к "веселой царице" пришла во время празднования великого праздника --- Рождество 25 декабря 1761г.
}
\end{document}
