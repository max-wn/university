% template - article
\documentclass[a4paper,12pt]{article} % format of document

\usepackage[english,russian]{babel} % add eng,rus(base) package
\usepackage[T1,T2A]{fontenc}        % add eng,rus encoding support
\usepackage[utf8]{inputenc}         % add UTF8 support
\usepackage{soul}         	    % add l a t e r s p a c i n g
\usepackage{longtable}         	    % To display tables on several pages
\usepackage{booktabs}         	    % For pretter tables
\usepackage{enumitem}         	    % advanced list support

\usepackage{amsmath, amsfonts, amssymb, amsthm, mathtools} % add math support

\usepackage{geometry}  % add document's fields correction support
\geometry{top=25mm}    % top field
\geometry{bottom=30mm} % bottom field
\geometry{left=20mm}   % left field
\geometry{right=20mm}  % right field

\linespread{1}               % length between str
\setlength{\parindent}{20pt} % red str
\setlength{\parskip}{12pt}   % length between paragraphs

\usepackage[backend=biber, style=authoryear-icomp]{biblatex} % add bibliography support
\addbibresource{$HOME/latex-templates/biblio.bib}            % path to bibliography base
\usepackage{csquotes}                                        % advanced facilities for inline and display quotations

\usepackage{indentfirst} % first paragraph with red str

% Must be the last command into the preamble of document.
\usepackage{hyperref} % All references in document turn into hyperlinks
\hypersetup{
unicode=true,      % юникод в названиях разделов pdf
colorlinks=true,   % цветные ссылки вместо ссылок в рамках
linkcolor=blue,    % внутренние ссылки
citecolor=green,   % ссылки на библиографию
filecolor=magneta, % ссылки на файлы
urlcolor=blue,     % ссылки на url
}
 % here is document's settings for russian
%\input{$HOME/latex-templates/preamble_article_eng.tex} % here is document's settings for english

\title{\so{MEIN REZEPTBUCH}}
\author{max-wn}
\date{April 12, 1961}

\begin{document}

\maketitle
\newpage

\large{Полёт первого человека в космическое пространство стал одним из знаковых событий ХХ века. Этот успех заметно упрочнил позиции Советского Союза на международной арене. Однако на сегодняшний день освещение истории формирования первого отряда космонафтов явно устарело. Больший интерес представляет то, как формировалась и по каким принципам работала система отбора в космонавты. Именно об этом пишет В.С. Батченко в своей статье "Первый набор в космонавты: от идеии к воплащению".}


В данной работе опысывается иследования на тему отбора людей для полета в космос. Началось все с секретного постановления Совета министров СССР 30 декабря 1949 года, "О дальнейшем развитии работ по исследованию верхних слоев атмосферы". Согласно ему в 1950-1951 году предусматривались подготовка и проведение серии запусков первой советской ракеты Р-1, сопроваждемых геофизическими и медико-биологическими исследованиями. Группа военных врачей Института авиамедицины занялась изучением влияния космических факторов на животных, в первую очередь собак. После успешног ополёта собаки Лайки (ноябрь 1957 год) для решения каким должен быть первый полет человека в космос, в ОКБ-1 создали две группы. одна должна было проработать баллистический полёт другая орбитальный.


\end{document}
