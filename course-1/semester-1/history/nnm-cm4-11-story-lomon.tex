\documentclass[a4paper,12pt]{article} % format of document

\usepackage[english,russian]{babel} % add eng,rus(base) package
\usepackage[T1,T2A]{fontenc}        % add eng,rus encoding support
\usepackage[utf8]{inputenc}         % add UTF8 support
\usepackage{soul}         	    % add l a t e r s p a c i n g
\usepackage{longtable}         	    % To display tables on several pages
\usepackage{booktabs}         	    % For pretter tables
\usepackage{enumitem}         	    % advanced list support

\usepackage{amsmath, amsfonts, amssymb, amsthm, mathtools} % add math support

\usepackage{geometry}  % add document's fields correction support
\geometry{top=25mm}    % top field
\geometry{bottom=30mm} % bottom field
\geometry{left=20mm}   % left field
\geometry{right=20mm}  % right field

\linespread{1}               % length between str
\setlength{\parindent}{20pt} % red str
\setlength{\parskip}{12pt}   % length between paragraphs

\usepackage[backend=biber, style=authoryear-icomp]{biblatex} % add bibliography support
\addbibresource{$HOME/latex-templates/biblio.bib}            % path to bibliography base
\usepackage{csquotes}                                        % advanced facilities for inline and display quotations

\usepackage{indentfirst} % first paragraph with red str

% Must be the last command into the preamble of document.
\usepackage{hyperref} % All references in document turn into hyperlinks
\hypersetup{
unicode=true,      % юникод в названиях разделов pdf
colorlinks=true,   % цветные ссылки вместо ссылок в рамках
linkcolor=blue,    % внутренние ссылки
citecolor=green,   % ссылки на библиографию
filecolor=magneta, % ссылки на файлы
urlcolor=blue,     % ссылки на url
}
 % here is document's settings for russian
%\input{$HOME/studyproject/universe/history/preamble-beamer-eng.tex} % here is document's settings for english


\title{Историческое сочинение по теме: М.В.Ломоносов --- выдающийся российский ученый}
\author{Немков Николай Максимович СМ4-11}

\date{16.12.2023}

\begin{document}

\maketitle
\begin{center}
Московски Государственный Технический Университет им. Н.Э. Баумана

Преподаватель: Щербакова Ольга Михайловна
\end{center}

\newpage

	Россия в начале 18 века была подвергнута Петровской модернизации. Царь-реформатор особое значение придавал науке, поскольку он твердо верил в то, что экономика и военное могущество государства находятся в неразрывной связи с развитием научного знания. Естественные науки стали государсвенной необходимостью. Возникла потребность в собственной академии наук, которая была утверждена в России в 1724г. по инициативе Петра I. Членами академии были исключительно приглашенные из Европы ученые. Для окончательного закрепления науки в России необходим был свой научный гений. Именно эта историческая роль и выпала на Михаила Васильевича Ломоносова - поморского мужика.

	Ломоносов родился 18 ноября 1711г. в деревне Мишанинской Куростровской волости Двинского уезда Архангелогородской губернии у черносошного крестьянина помора Василия Дорофеевича Ломоносова и его жены Елены Ивановны. В 19 лет он получил паспорт в Холмогорской воеводской канцелярии и с рыбным обозом отправился в Москву.

	Юноша еще в Холмогорах получил хорошее образование, даже владел несколькими иностранными языками. Приехав в Москву, он поступил в Свято-греко-латинскую академию, где очень выделялся среди учеников. Судьбоносную роль сыграло то, что М.В.Ломоносов оказался среди 12 учеников для продолжения учебы в Академическом университете в Петербурге. Отсюда вскоре вместе с Д.И.Виноградовым и Г.У.Райзером был отправлен в Германию для обучения химии и горному делу.

	Годы обучения в Марбургском университете оказались очень благотворными для становления Ломоносова как ученого-естествоиспытателя. Здесь же он написал первою свою студенческую диссертацию "Работа по физике о превращении твердого тела в жидкое в зависимости от движения предшествующей жидкости " 4 октября 1738г. После Марбурга недолго обучался горному делу у академика И.Ф.Генкеля. Ссора с Генкелем сделала Ломоносова отцом практической геологии и металлургии в России.

	В 1741г. Михаил Васильевич вернулся в Россию, где в полной мере раскрылась его научная мощь и организаторские способности.

	Открытия Ломоносова в области химии(цветное стекло, физическая химия), в области физики(молекулярно-кинетическая теория тепла, молекулы корпускулы), в области астрономии и метеорологии(открытие атмосферы у Венеры, атмосферное электричество, многослойность атмосферы), в области географии и геологии(классификация льдов, теория о строении земли).

	М.В.Ломоносов стремительно достиг вершин мировой науки, его труды составили целую эпоху в развитии русской науки, Петербургской Академии наук. Известны слова А.С.Пушкина "...Он создал первый университет; он лучше сказать, сам был первым университетом".

\end{document}
