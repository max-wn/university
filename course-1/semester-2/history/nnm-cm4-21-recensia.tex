% template - article
\documentclass[a4paper,12pt]{article} % format of document

\usepackage[english,russian]{babel} % add eng,rus(base) package
\usepackage[T1,T2A]{fontenc}        % add eng,rus encoding support
\usepackage[utf8]{inputenc}         % add UTF8 support
\usepackage{soul}         	    % add l a t e r s p a c i n g
\usepackage{longtable}         	    % To display tables on several pages
\usepackage{booktabs}         	    % For pretter tables
\usepackage{enumitem}         	    % advanced list support

\usepackage{amsmath, amsfonts, amssymb, amsthm, mathtools} % add math support

\usepackage{geometry}  % add document's fields correction support
\geometry{top=25mm}    % top field
\geometry{bottom=30mm} % bottom field
\geometry{left=20mm}   % left field
\geometry{right=20mm}  % right field

\linespread{1}               % length between str
\setlength{\parindent}{20pt} % red str
\setlength{\parskip}{12pt}   % length between paragraphs

\usepackage[backend=biber, style=authoryear-icomp]{biblatex} % add bibliography support
\addbibresource{$HOME/latex-templates/biblio.bib}            % path to bibliography base
\usepackage{csquotes}                                        % advanced facilities for inline and display quotations

\usepackage{indentfirst} % first paragraph with red str

% Must be the last command into the preamble of document.
\usepackage{hyperref} % All references in document turn into hyperlinks
\hypersetup{
unicode=true,      % юникод в названиях разделов pdf
colorlinks=true,   % цветные ссылки вместо ссылок в рамках
linkcolor=blue,    % внутренние ссылки
citecolor=green,   % ссылки на библиографию
filecolor=magneta, % ссылки на файлы
urlcolor=blue,     % ссылки на url
}
 % here is document's settings for russian
%\input{$HOME/latex-templates/preamble_article_eng.tex} % here is document's settings for english

\title{\so{MEIN REZEPTBUCH}}
\author{max-wn}
\date{April 12, 1961}

\begin{document}


\large{Полёт первого человека в космическое пространство стал одним из знаковых событий ХХ века. Этот успех заметно упрочнил позиции Советского Союза на международной арене. Однако на сегодняшний день освещение истории формирования первого отряда космонафтов явно устарело. Больший интерес представляет то, как формировалась и по каким принципам работала система отбора в космонавты. Именно об этом пишет В.С. Батченко в своей статье "Первый набор в космонавты: от идеии к воплащению".}


В данной работе опысывается иследования на тему отбора людей для полета в космос. Началось все с секретного постановления Совета министров СССР 30 декабря 1949 года, "О дальнейшем развитии работ по исследованию верхних слоев атмосферы". Согласно ему в 1950-1951 году предусматривались подготовка и проведение серии запусков первой советской ракеты Р-1, сопроваждемых геофизическими и медико-биологическими исследованиями. Группа военных врачей Института авиамедицины занялась изучением влияния космических факторов на животных, в первую очередь собак. После успешного полёта собаки Лайки (ноябрь 1957 год) для решения каким должен быть первый полет человека в космос, в ОКБ-1 создали две группы. одна должна было проработать баллистический полёт другая орбитальный. Было принято решение о целесообразности орбитального полета. К тому времени учёные уже имели налаженную систему спасения живых су-
ществ из суборбитальных полётов.

Тем временем 20 июня Совет национальной безопасности США утвердил
стратегию космической политики страны, выделив пилотируемые полёты в от-
дельное направление деятельности. Советскому Союзу это грозило потерей
позиций в освоении космического пространства. Для начала отбора и подго-
товки кандидатов на осуществление полётов требовалось не только одобрение
органов власти, но и существенное увеличение материальных ресурсов и штата
НИИИАМ – единственной на тот момент организации, проводившей исследования в области космической медицины.

В СССР 5 января того же года вышло секретное постановление ЦК КПСС
и Совета министров СССР № 22–10 "Об усилении научно-­исследовательских
работ в области медико-­биологического обеспечения космических полётов".
Институт авиа-медицины преобразовали в Государственный НИИ авиационной и космической медицины с увеличением штата на 100 человек и созданием необходимой
материально-­технической базы.
Следом состоялось совещание под председательством М.В.Келдыша, "на котором подробно обсуждался вопрос о полёте в космос человека, вплоть до того, из кого выбирать будущих кандидатов в космонавты". Приняв за основу, что отбор следует
проводить среди лётчиков истребительной авиации, участники совещания пришли к выводу, что разработку принципов отбора и план действий необходимо
поручить авиационным врачам. На вооружение взяли опыт работы Центральной врачебно-летной комиссии Минобороны СССР и опыт американцев приступивших к отбору немногим ранее.


Для координации действий военных врачей к лету 1959 г. составили "Ин-
струкцию для членов врачебных комиссий по отбору кандидатов в космонавты
в воинских частях". Разработка принципов отбора осуществлялась на основе данных авиационной
медицины и с привлечением последних достижений психологии.
Инструкция зафиксировала антропометрические требования к будущим космонавтам:
"физически здоровые лётчики в возрасте не старше 35 лет, ростом в преде-
лах 165–175 см и весом не более 75 кг". Кроме того учитывались личностные характеристики:
лётные навыки, поведение в сложных или аварийных ситуациях, моральная
и психологическая устойчивость, общественно-­политическая сознательность.
Главную роль при первичном отборе играли профессиональная подготовка
и личностные характеристики. Большинство кандидатов являлись лётчиками-­истребителями третьего класса


Завершил первый этап отбора и положил начало второму – госпитальному – совместный приказ главкома ВВС и начальника Главного военно-­медицинского управления № 00240 от 30 сентября 1959 г. о создании Главной
медицинской комиссии, "в задачу которой входило вынесение окончательного
экспертного медицинского заключения по результатам стационарного обследования кандидатов в космонавты"

Автор ясно и логично изложил всю информацию взятую из оффициальных рассекреченных документов.
В данной статье была приведена борьба двух держав за первенство в космическом пространстве, также были статистические данные
это показывает объективнось Батченко.


Хоть в тексте было много цифр и статистики, было интересно читать об образовании системы отбора космонавтов, также присутствовали интересные цитаты, например группа отбора Гуровского на вопрос: «можно ли посоветоваться с супругой, можно ли подумать, дать
ответ на следующий день?» разрешалось лишь «походить по коридору» в одиночных раздумьях


Подводя итог, хочется сказать, что статья рассказала о многих аспектах выбора членов в отряд космонавтов и как это все зарождалось. После ознакомления с этом тектом однозначно расширяется общий кругозор. Я думаю после прочтения этой статьи, человек который не увлекается космонавтикой точно захочет узнать про эту тему побольше. Таким образом это статья -- познавательная, но всеже не полная. Как сказал сам автор - "Необходимо отметить, что, несмотря на активное рассекречивание архивных материалов, связанных с космонавтикой, до
сих пор учёные имеют свободный доступ не ко всем из них."


\begin{figure}[b]
	\Large{\textbf{Список литературы:}}

		1. В.С. Батченко: Первый набор в космонавты: от идеи к воплощению. Российская история. 2023. № 1

\end{figure}

\end{document}
